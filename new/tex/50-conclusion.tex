\anonsection{ЗАКЛЮЧЕНИЕ}
Была проанализирована предметная область, а также рассмотрены основные алгоритмы борьбы с шумами в изображениях.
Приведены критерии сравнения алгоритмов и осуществлена их классификация.

Полученные результаты можно свести к следующим тезисам:
\begin{itemize}
	\item шум на изображениях возникает вследствие несовершенства технических средств. На практике применяются алгоритмические средства борьбы;
	\item существующие алгоритмы, не использующие нейронные сети, справляются лишь с определенным видом шумов, имея проблемы с устранением остальных типов;
	\item методы, использующие сверточные нейронные сети, работают лишь с определенным размером входного изображения, при этом не производя сглаживание;
	\item в алгоритмах DnCNN и RIDNet детали реализации сети могут быть изменены: функция активации, количество слоев;
	\item новые решения в области удаления шумов являются лишь улучшением или комбинированием уже существующих алгоритмов.
\end{itemize}