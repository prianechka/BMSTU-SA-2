\anonsection{ВВЕДЕНИЕ}
Цифровой шум в изображениях появляется вследствие свойств фотона света или вмешательства человека.
Он оказывает огромное влияние на восприятие человеком качества картинки, а также ухудшает результаты работы алгоритмов, анализирующих изображения. 
Нейросети плохо справляются с определением объектов на фотографии, если внести на неё шум, причём человеческий глаз не всегда в состоянии распознать, что на картинке присутствуют помехи.

\textbf{Актуальность работы} состоит в том, что аппаратные способы борьбы с шумами в изображениях в текущий момент не реализованы, и компании стараются бороться с дефектами на фотографиях с помощью алгоритмических методов. 
Например, Google в линейке телефонов Pixel используют алгоритмы для улучшения качества фотосъемки.
Технические средства Kodak обрабатывают изображения с помощью таких методов, как медианный фильтр.

\textbf{Цель работы} --- проанализировать существующие методы борьбы с шумами в изображениях.

Для достижения поставленной цели требуется решить \textbf{задачи}:
\begin{itemize}
\item описать термин шума, причины его появления и классифицировать его типы;
\item описать существующие методы обнаружения шума в изображениях и способы его устранения;
\item сформулировать критерии сравнения рассмотренных методов;
\item классифицировать существующие алгоритмы борьбы с шумами.
\end{itemize}
