\section*{ВВЕДЕНИЕ}
\addcontentsline{toc}{section}{ВВЕДЕНИЕ}
Цифровой шум в изображениях появляется вследствие физических свойств фотона света или вмешательства человека.
Он оказывает огромное влияние на восприятие человеком качества картинки, а также ухудшает результаты работы алгоритмов, анализирующих изображения. 
Нейросети плохо справляются с определением объектов на фотографии, если внести на неё шум, причём человеческий глаз не всегда в состоянии распознать, что на картинке присутствуют помехи.

Актуальность работы состоит в том, что многие существующие алгоритмы предполагают вмешательство человека в процесс шумоподавления, в то время как этот процесс может быть автоматизирован.

Цель работы --- проанализировать существующие методы борьбы с шумами в изображениях.

Для достижения поставленной цели потребуется:
\begin{itemize}
\item Описать термин шума, причины его появления и классифицировать его типы;
\item Описать существующие методы обнаружения шума в изображениях и способы его устранения;
\item Сформулировать критерии сравнения рассмотренных методов;
\item Классифицировать существующие алгоритмы борьбы с шумами.
\end{itemize}
