%----------------------- Преамбула -----------------------
\documentclass[ut8x, 14pt, oneside, a4paper]{extarticle}

\usepackage{extsizes} % Для добавления в параметры класса документа 14pt

% Для работы с несколькими языками и шрифтом Times New Roman по-умолчанию
\usepackage{cmap} % Улучшенный поиск русских слов в полученном pdf-файле
\usepackage{graphicx}
\usepackage[english,russian]{babel}
\usepackage[utf8]{inputenc} % Кодировка utf8
\usepackage{fontspec}
\setmainfont{Times New Roman}
\usepackage{float}
\usepackage{pdfpages}
\usepackage{amsmath}

% ГОСТовские настройки для полей и абзацев
\usepackage{geometry}
\geometry{left=30mm}
\geometry{right=10mm}
\geometry{top=20mm}
\geometry{bottom=20mm}
\usepackage{misccorr}
\usepackage{indentfirst}
\usepackage{enumitem}
\setlength{\parindent}{1.25cm}
\renewcommand{\baselinestretch}{1.25}
\setlist{nolistsep} % Отсутствие отступов между элементами \enumerate и \itemize


% Переопределение стандартных \section, \subsection, \subsubsection по ГОСТу;
% Переопределение их отступов до и после для 1.5 интервала во всем документе
\usepackage{titlesec}

\titleformat{\subsection}[hang]
{\bfseries\normalsize}{\thesubsection}{1em}{}
\titlespacing\subsection{\parindent}{\parskip}{\parskip}

\titleformat{\subsubsection}[hang]
{\bfseries\normalsize}{\thesubsubsection}{1em}{}
\titlespacing\subsubsection{\parindent}{\parskip}{\parskip}

% Работа с изображениями и таблицами; переопределение названий по ГОСТу
\usepackage{float}
\usepackage{setspace}
\onehalfspacing % Полуторный интервал
\frenchspacing
\usepackage{indentfirst} % Красная строка
\titleformat{\chapter}{\LARGE\bfseries}{\thechapter}{20pt}{\LARGE\bfseries}
\titleformat{\section}{\centering\Large\bfseries}{\thesection\quad}{20pt}{\Large\bfseries}
\usepackage{listings}

% Выравнивание "где" по ГОСТу для формул
\usepackage{eqexpl}


% Цвета для гиперссылок и листингов
\usepackage{color}

% Для нормальной нумерации картинок, разделённой по section'ам
\usepackage{chngcntr}
\counterwithin{figure}{section}
\counterwithin{equation}{section}

% Гиперссылки \toc с кликабельностью
\usepackage{hyperref}

\hypersetup{
	linktoc=all,
	linkcolor=black,
	colorlinks=true,
}

% Листинги
\setsansfont{Arial}
\setmonofont{Courier New}

\usepackage{color}
\hypersetup{citecolor=black}

\usepackage{ulem} % Нормальное нижнее подчеркивание
\usepackage{hhline} % Двойная горизонтальная линия в таблицах
\usepackage[figure,table]{totalcount} % Подсчет изображений, таблиц
\usepackage{rotating} % Поворот изображения вместе с названием
\usepackage{lastpage} % Для подсчета числа страниц

\makeatletter
\renewcommand\@biblabel[1]{#1.}
\makeatother

\usepackage{color}
\usepackage[cache=false, newfloat]{minted}
\newenvironment{code}{\captionsetup{type=listing}}{}
\SetupFloatingEnvironment{listing}{name=Листинг}


\newcommand{\anonsection}[1]{%
	\section*{\centering#1}%
	\addcontentsline{toc}{section}{#1}%
}


\makeatletter
\renewenvironment{thebibliography}[1]
{\section*{\bibname}% <-- this line was changed from \chapter* to \section*
	\@mkboth{\MakeUppercase\bibname}{\MakeUppercase\bibname}%
	\list{\@biblabel{\@arabic\c@enumiv}}%
	{\settowidth\labelwidth{\@biblabel{#1}}%
		\leftmargin\labelwidth
		\advance\leftmargin\labelsep
		\@openbib@code
		\usecounter{enumiv}%
		\let\p@enumiv\@empty
		\renewcommand\theenumiv{\@arabic\c@enumiv}}%
	\sloppy
	\clubpenalty4000
	\@clubpenalty \clubpenalty
	\widowpenalty4000%
	\sfcode`\.\@m}
{\def\@noitemerr
	{\@latex@warning{Empty `thebibliography' environment}}%
	\endlist}
\makeatother

\renewcommand\labelitemi{---}
\usepackage[figurename=Рисунок, labelsep=endash, singlelinecheck=false]{caption}
\usepackage{placeins}
\usepackage{ragged2e}
\justifying
\captionsetup[table]{justification=raggedleft,singlelinecheck=off}

